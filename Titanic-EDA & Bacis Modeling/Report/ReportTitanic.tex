\documentclass[11pt]{article}

\usepackage{amsmath, amssymb}
\usepackage{xepersian}
\settextfont{Yas}
\setmathdigitfont{Yas}
\deflatinfont\Consolas{Consolas}
\linespread{1.2}
\title{گزارش کار بخش پیاده سازی تمرین دوم درس یادگیری ماشین}
\author{محمد لشکری}
\begin{document}
	\maketitle
مجموعه متغیر‌های مستقل و وابسته به صورت زیر انتخاب شده‌اند:
\lr{
\begin{description}
	\item
	{\Consolas 	X = \{Pclass, Sex, Age, SibSp, Parch, Fare, Embarked\}}
	\item
	{\Consolas Y = \{Survived\}}
\end{description}
}
همانطور که مشاهده می‌شود بعضی از ویژگی‌ها مانند شناسه مسافر و بلیت و کابین به دلیل یکتا بودن مقادیر حذف شده‌اند.
\section{پیش پردازش داده‌ها}\label{preprocessing}
با استفاده از
{\Consolas KNNImpuer}
با در نظر گرفتن ۲۰ همسایه مقادیر گمشده در ویژگی های سن و کرایه بلیت جایگزین شده‌اند. ویژگی های جنسیت و 
{\Consolas Embarked}
به نوع عددی تغییر پیدا کردند. نسبت دادن بازه به داده در ویژگی‌های سن و کرایه بلیت باعث حذف اطلاعات از دادگان می‌شود اما چون دامنه این متغیر‌ها گسترده است با استفاده از مقیاس کننده استاندارد آنها را مقیاس کرده‌ایم تا عددی در بازه 
$\left[-3,3\right]$
اختیار کنند. لازم یه ذکر است منفی شدن اعداد در نتیجه تأثیر ندارد.
\section{مدل‌سازی}\label{modeling}
قبل از فرایند مدل‌سازی، دادگان آموزش به دو بخش آموزش و تست با نسبت ۰/۲ تقسیم  شدند. سه الگوریتم زیر روی دادگان اموزش دیده‌اند:
\lr{
\begin{itemize}
	\item Support Vector Machine (SVM)
	\item Logistic Regression (LR)
	\item Gaussian Naïve bayes (GNB)
\end{itemize}
}
که درآن کرنل
 \lr{SVM}
 	خطی است.
نتایج زیر با استفاده از 
	\lr{cross validation}
	با تعداد ۵ فلدر به دست آمده است:
\begin{latin}
	\begin{table}[h]
		\parbox{\linewidth}{
		\begin{center}
		\begin{tabular}{|c|c|c|c|}
			\hline
			& precision(macro ave.)&precision of class 0(Test set)& precision of class 1(Test set)\\
			\hline
			SVM & 0.78 & 0.84 & 0.70\\
			LR & 0.79 & 0.87 & 0.74\\
			GNB & 0.78 & 0.78 & 0.77\\
			\hline
		\end{tabular}
		\end{center}
		}
		\parbox{\linewidth}{
		\begin{center}
		\begin{tabular}{|c|c|c|c|}
		\hline
		& recall(macro ave.) & recall f class 0(Test set) & recall of class 1(Test set)\\
		\hline
		SVM & 0.76 & 0.80 & 0.75\\
		LR & 0.77 & 0.83 & 0.80\\
		GNB & 0.78 & 0.83 & 0.71\\
		\hline
		\end{tabular}
		\end{center}
		}
		\parbox{\linewidth}{
		\begin{center}
		\begin{tabular}{|c|c|c|c|}
		\hline
		& f1-score(macro ave.) & f1-score of class 0(Test set) & f1-score of class 1(Test set)\\
		\hline
		SVM & 0.77 & 0.82 & 0.73 \\
		LR & 0.78 & 0.85 & 0.77 \\
		GNB & 0.78 & 0.80 & 0.74 \\
		\hline
		\end{tabular}
		\end{center}
		}
		\parbox{0.1\linewidth}{
		\begin{center}
			\begin{tabular}{|c|c|c|}
				\hline
				& accuracy(CV)& accuracy(Test set) \\
				\hline
				SVM & 0.79 & 0.78\\
				LR & 0.80  & 0.82\\
				GNB & 0.79 & 0.78\\
				\hline
			\end{tabular}
		\end{center}			
		}
	\end{table}
\end{latin}

پیش بینی درست زنده ماندن یا نماندن افراد در این مسأله اهمیت بیشتری دارد. پس مدلی که بالاترین مقدار \lr{precision} را دارد بهتر است. همانطور که مشاهده می‌شود مقدار این کمیت برای 
\lr{CV}
 در مدل
\lr{LR}
بالاتر است و برای داده‌های تست نیز نزدیک به سایرین است. همچنین دقت عمومی  
\lr{(accuracy)}
برای این مدل بالاتر از مدل‌های دیگر است. پس این مدل، از دو مدل دیگر عملکرد بهتری داشته است. در فایل کد مجموعه دادگان موجود در 
\lr{test.csv}
با 
\lr{\Consolas X\textunderscore submission}
نمایش داده شده و نتایج مربوط به آن، در فایل 
\lr{gender\textunderscore submission.csv}
قرار گرفته است.
\end{document}	